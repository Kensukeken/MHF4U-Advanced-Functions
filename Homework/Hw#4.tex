\documentclass{article}
\usepackage[utf8]{inputenc}
\usepackage[T1]{fontenc}
\usepackage{graphicx}
\usepackage{amsmath, amssymb}
\usepackage{xcolor}
\usepackage{tikz}
\usepackage{lipsum}
\usepackage{sectsty}

\title{Practice Questions}
\author{Hia Al Saleh}
\date{February 29th, 2024}
\begin{document}
\maketitle
\section*{Chapter 1.7 \& Review}
Page 71, \#1, 2, 4, 6, 7, 9, 11, 12
\begin{itemize}
\item Question 1:
\begin{enumerate}
    \item \textbf{Part A:} The coordinates of the tangent point are (5, 3), as indicated by the graph.
    
    \item \textbf{Part B:} To state the coordinates of another point on the tangent line, we can choose any point that lies on the tangent line. Let's pick a point close to (5, 3) on the graph, such as (3, 7).
    
    \item \textbf{Part C:} Using the points from Parts A and B, we can calculate the slope of the tangent line. The slope is given by the change in y-coordinates divided by the change in x-coordinates, which is \(\frac{3 - 7}{5 - 3} = \frac{-4}{2} = -2\).
    
    \item \textbf{Part D:} The slope of -2 at the point (5, 3) indicates that at that instant moment on the curve, the curve is decreasing at a rate of 2 units in the y-direction for every 1 unit in the x-direction.
\end{enumerate}

\item Question 2:
2a)
\begin{enumerate}
    \item The instantaneous rate of change at point \(A\): If you draw a tangent line to pass through that point, you can see the slope is very positive. Therefore, we say the instantaneous rate of change at \(A\) is positive.
    
    \item The instantaneous rate of change at point \(B\): Its horizontal slope is zero, so the instantaneous rate of change is zero.
    
    \item The instantaneous rate of change at point \(C\): You can see it has a very negative slope, so the rate of change is negative.
\end{enumerate}
2b)
\begin{enumerate}
    \item \textbf{At Point A:} The slope of the tangent line appears to be positive and quite steep, indicating a high rate of change. This means the tennis ball is moving upwards quickly.
    
    \item \textbf{At Point C:} The slope of the tangent line is also positive but less steep than at point A, suggesting the ball is still moving upwards but at a slower rate than at point A.
\end{enumerate}
2c)
The interpretation of these values in the context of the situation represented by the graph would be:

\begin{enumerate}
    \item \textbf{At Point A:} The positive rate of change indicates the ball is ascending rapidly after being hit.
    
    \item \textbf{At Point C:} The positive rate of change, though smaller than at point A, indicates the ball is nearing the peak of its trajectory and will soon start descending.
\end{enumerate}

\item Question 4:
\begin{enumerate}
    \item To estimate the slope of the tangent line, we can use any two points on the line. For example, let's use the point with coordinates (1.5, 45) and another point that seems to be close to (3, 30).
    
    \item The change in \(x\) (\(\Delta x\)) is \(3 - 1.5 = 1.5\) and the change in \(y\) (\(\Delta y\)) is \(30 - 45 = -15\). So, the estimated slope will be \(\frac{-15}{1.5} = -10\) meters per second.
    
    \item Alternatively, we could use another pair of points for estimation. Let's say one point is (1.5, 45) and another point close to (3, 67.5).
    
    \item In this case, \(\Delta x = 3 - 1.5 = 1.5\) and \(\Delta y = 67.5 - 45 = 22.5\). So, the estimated slope will be \(\frac{22.5}{1.5} = 15\) meters per second.
    
    \item It's important to note that these values are just estimations, as the exact points on the line are not known. The choice of points for estimation can lead to variations in the calculated slope.
\end{enumerate}

\item Question 6:
\begin{enumerate}
    \item Determine the average rate of change from 1955 to 2005. This involves finding the slope between the ordered pairs (1955, 16.8) and (2005, 127.3). The average rate of change is given by \(\frac{\Delta \text{CPI}}{\Delta \text{years}}\), which equals \(\frac{127.3 - 16.8}{2005 - 1955} = \frac{110.5}{50} = 2.21\).
    
    \item Estimate the instantaneous rate of change for CPI in 1965. To estimate this, we can use the points (1960, 18.5) and (1970, 24.2). The estimated slope is \(\frac{24.2 - 18.5}{1970 - 1960} = \frac{5.7}{10} = 0.57\).
    
    \item Similarly, estimate the average rate of change for 1985. Using the points (1980, 52.4) and (1990, 93.3), the estimated slope is \(\frac{93.3 - 52.4}{1990 - 1980} = \frac{40.9}{10} = 4.09\).
    
    \item Lastly, estimate the instantaneous rate of change at 2000. Using the points (1995, 87.6) and (2005, 127.3), the estimated slope is \(\frac{127.3 - 87.6}{2005 - 1995} = \frac{39.7}{10} = 3.97\).
    
    \item The overall average rate of change suggests that initially, the rate of change was slow, but it increased significantly and then stabilized towards the end of the period.
\end{enumerate}

\item Question 7:
\begin{enumerate}
    \item[7a)] The soccer ball is kicked into the air and is modeled by the function \( h(t) \). Determine the average rate of change of the height of the ball from one to two seconds. The average rate of change will be given by:

\[
\frac{{\text{{height at two seconds}} - \text{{height at one second}}}}{{\text{{difference in time}}}}
\]

So, it's going to be:

\[
\frac{{-4.9 \times 2^2 + 12 \times 2 + 0.5 - (-4.9 \times 1^2 + 12 \times 1 + 0.5)}}{{2 - 1}}
\]

Simplify this expression:

\[
= \frac{{-4.9(2^2 - 1) + 12(2 - 1)}}{{1}} = \frac{{-4.9 \times 3 + 12}}{{1}} = \frac{{-14.7 + 12}}{{1}} = \frac{{-2.7}}{{1}} = -2.7 \text{ meters per second}
\]
 \item[7b)] It's an instantaneous rate of change problem. The soccer balls kick into the air such that height \( h \) in meters after \( t \) seconds is modeled with this. Okay, what is the instantaneous rate of change after one second?

So, that's \( h \) at, let's say, \( 1.001 \) minus \( h \) at \( 1 \).

And then if we take the difference between \( 1.001 \) and \( 1 \), this will give you \( 0.001 \).

This is basically if you have a parabola. Okay, here you want to find the instantaneous rate of change, the instant slope there at one. I'm just taking a point really close to it and find a slope between these two really close adjacent points to estimate the slope, okay? All right, so that means we're going to plug this in here:

\[ \frac{{1.001^2 + 12 \times 1.001 + 0.5 - (1.001^2 + 12 \times 1 + 0.5)}}{{0.001}} \]

So, this cancels out and I get \( 2.1951 \).

\end{enumerate}
\item Question 9:
\textbf{(a) Average rate of change of the cost from producing 100 to 200 MP3 players:}

Given the cost function \( C(x) = 0.00015x^3 + 100x \), we need to find:

\[ \text{Average rate of change} = \frac{C(200) - C(100)}{200 - 100} \]

\[ C(200) = 0.00015(200)^3 + 100(200) = 12000 \]

\[ C(100) = 0.00015(100)^3 + 100(100) = 11500 \]

\[ \text{Average rate of change} = \frac{12000 - 11500}{200 - 100} = \frac{500}{100} = 5 \]

So, the average rate of change of the cost from producing 100 to 200 MP3 players is \$5 per player.

\textbf{(b) Instantaneous rate of change of the cost for producing 200 MP3 players:}

To find the instantaneous rate of change, we need to find the derivative of the cost function \( C(x) \) with respect to \( x \), denoted as \( C'(x) \), and then evaluate it at \( x = 200 \).

\[ C'(x) = \frac{d}{dx}(0.00015x^3 + 100x) = 0.00045x^2 + 100 \]

\[ C'(200) = 0.00045(200)^2 + 100 = 180 \]

So, the instantaneous rate of change of the cost for producing 200 MP3 players is \$180 per player.

\textbf{(c) Interpretation:}

The average rate of change tells us that, on average, the cost increases by \$5 for each additional MP3 player produced between 100 and 200 players.

The instantaneous rate of change tells us that, at the specific production level of 200 MP3 players, the cost increases by \$180 for each additional player produced.

\textbf{(d) Does the cost ever decrease?}

To determine if the cost ever decreases, we need to examine the sign of the derivative \( C'(x) \). If \( C'(x) \) is negative for some values of \( x \), then the cost is decreasing. Otherwise, if \( C'(x) \) is always positive or zero, then the cost is always increasing.

\[ C'(x) = 0.00045x^2 + 100 \]

This is always positive because the quadratic term is always positive. Therefore, the cost never decreases; it always increases as production increases.

\item Question 11:
\textbf{1. Average Rate of Change:}

\[ \text{Average rate of change} = \frac{C(200) - C(100)}{200 - 100} \]

\[ C(200) = 0.00015(200)^3 + 100(200) = 12000 \]

\[ C(100) = 0.00015(100)^3 + 100(100) = 11500 \]

\[ \text{Average rate of change} = \frac{12000 - 11500}{200 - 100} = \frac{500}{100} = 5 \]

\textbf{2. Instantaneous Rate of Change:}

\[ C'(200) = 0.00045(200)^2 + 100 = 180 \]

\textbf{3. Interpretation of Values:}

- The average rate of change of 5 per player indicates that, on average, the cost increases by \$5 for each additional MP3 player produced between 100 and 200 units.

- The instantaneous rate of change of 180 per player tells us that, at the specific production level of 200 MP3 players, the cost increases by \$180 for each additional player produced.

\textbf{4. Cost Decrease:}

To determine if the cost ever decreases, we examine the sign of the derivative \(C'(x)\). Since \(C'(x) = 0.00045x^2 + 100\) is always positive for positive values of \(x\), the cost does not decrease as production increases.

\item Question 12:
\begin{enumerate}
    \item[12a)] 
The population \( P \) of a small town after \( T \) years can be modeled with the function \( P(T) \), where \( T = 0 \) represents the beginning of this year. 

To find the average rate of change of \( P \) with respect to \( T \) from \( T = 0 \) to \( T = 8 \), we use the formula for the average rate of change:

\[ \text{Average rate of change} = \frac{P(8) - P(0)}{8 - 0} \]

Simplifying, we get:

\[ \text{Average rate of change} = \frac{P(8) - P(0)}{8} \]

Now, let's express \( P(8) - P(0) \) as \( P(8 + h) - P(h) \), where \( h \) is a small increment. So, the expression for the average rate of change becomes:

\[ \text{Average rate of change} = \frac{P(8 + h) - P(h)}{8} \]

To further simplify, we use the limit definition of the derivative:

\[
\text{Average rate of change} = \lim_{h \to 0} \frac{P(8 + h) - P(h)}{h}
\]

\[
= \lim_{h \to 0} \frac{-0.09(8 + h)^3 + 1.89(8 + h)^2 + 9(8 + h) + 0.09h^3 - 1.89h^2 - 9h}{h}
\]

\[
= \lim_{h \to 0} \frac{-0.09(512 + 192h + 24h^2 + h^3) + 1.89(64 + 16h + h^2) + 72 + 9h + 0.09h^3 - 1.89h^2 - 9h}{h}
\]

\[
= \lim_{h \to 0} \frac{(-0.09 - 0.09)h^3 + (24 - 1.89 - 1.89)h^2 + (192 + 16 - 9 - 9)h}{h}
\]

\[
= \lim_{h \to 0} \frac{h(-0.18h^2 + 20.22h + 190)}{h} = \lim_{h \to 0} (-0.18h^2 + 20.22h + 190)
\]

\[
= -0.18(0)^2 + 20.22(0) + 190 = 190
\]

Therefore, the average rate of change is 190. 



In our calculation, \( P(8 + h) - P(h) \) simplifies to \( 0.5h \). Therefore, the expression becomes:

\[ \text{Average rate of change} = \lim_{h \to 0} \frac{0.5h}{h} \]

\[ = \lim_{h \to 0} 0.5 \]

\[ = 0.5 \]

So, the average rate of change of the population from \( T = 0 \) to \( T = 8 \) is 0.5.

\item[12bc)] We have to use what we got from Part A to find the average rate of change when \( h \) is equal to two, so two years from the eighth year, right? So, two to eighth to tenth year, eight to twelve years, and eight to fifteen years. Let's plug it in the average rate of change:

\textbf{Average rate of change when \( T = 2 \):}
\[ \text{Average rate of change} = \frac{P(10) - P(8)}{10 - 8} = \frac{(8 + 10)^2 - 8^2}{10 - 8} = \frac{272}{2} = 136 \text{ per year} \]

So, an average of 136 people were added to the population per year.

\textbf{Average rate of change when \( T = 4 \):}
\[ \text{Average rate of change} = \frac{P(12) - P(8)}{12 - 8} = \frac{(8 + 12)^2 - 8^2}{12 - 8} = \frac{302}{4} = 75.5 \text{ per year} \]

So, an average of 75.5 people were added to the population per year.

\textbf{Average rate of change when \( T = 5 \):}
\[ \text{Average rate of change} = \frac{P(13) - P(8)}{13 - 8} = \frac{(8 + 13)^2 - 8^2}{13 - 8} = \frac{318.5}{5} = 63.7 \text{ per year} \]

So, an average of 63.7 people were added to the population per year.

What this means is that the rate of change is increasing from year 2 to 4 to 5 because the overall average is increasing. So, that means the likely instantaneous rate of change at 4 and 5 is higher as we go farther down the years.
    \item[12de] Part D is really just asking us to find the instantaneous rate of change when the eighth year. Instead of incrementing \( h \) to be two, three, four years afterwards, let's just pick \( h \) to be a really small value. The smaller it is, the more accurate it's going to be. So, let's do the work here. Let's choose \( h \) to be \( 0.001 \). 

We already did all the simplifications earlier. Now, we're going to replace \( h \) with \( 0.001 \). 

\[ \text{Instantaneous rate of change} = 0.5 \times 8^2 + 8 \times 8 + 150 \]
\[ = 2046 \]

So, the instantaneous rate of change on the eighth year is that 2046 people were added in that year.

\end{enumerate}
\end{itemize}
\end{document}

