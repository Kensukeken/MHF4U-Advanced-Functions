\documentclass{article}
\usepackage[utf8]{inputenc}
\usepackage[T1]{fontenc}
\usepackage{graphicx}
\usepackage{amsmath, amssymb}
\usepackage{xcolor}
\usepackage{tikz}
\usepackage{enumitem}
\usepackage{lipsum}
\usetikzlibrary{fit}
\usepackage{hyperref}
\usepackage{subfig}
\usepackage{polynom}
\usepackage{pgfplots}
\usepackage{soul}
\usepackage{framed}
\usepackage[most]{tcolorbox}

% Define colors
\definecolor{lessoncolor}{RGB}{74, 144, 226}
\definecolor{examplecolor}{RGB}{92, 184, 92}
\definecolor{notecolor}{RGB}{255, 179, 102}

% Define custom environments
\newtcolorbox{lessonbox}[1]{
  colback=lessoncolor!5!white,
  colframe=lessoncolor!75!black,
  title=#1
}

\newtcolorbox{examplebox}[1]{
  colback=examplecolor!5!white,
  colframe=examplecolor!75!black,
  title=#1
}

\newtcolorbox{notebox}[1]{
  colback=notecolor!5!white,
  colframe=notecolor!75!black,
  title=#1
}

\begin{document}

\begin{titlepage}
    \centering
    \vspace*{2cm}
    {\LARGE \textcolor{lessoncolor}{Advanced Functions}}\par
    \vspace{1cm}
    {\large Kensukeken}\par
    \vspace{2cm}
    {\large May 9th, 2024}\par
    \vspace{3cm}
\end{titlepage}
\tableofcontents
\newpage
\section{Unit 6}
\section{Log and Exponent Functions and Equations}
\subsection{Logarithms}

\begin{lessonbox}{Introduction to Logarithms}
As we saw last class, the logarithm function is the inverse of an exponential. But what exactly does this mean?

\[
\begin{aligned}
    &\text{If } f(x) = 2^x \text{ then } f^{-1}(x) = \log_2(x)\\
    &\text{The function returns the value of a power when given an exponent.}\\
    &\text{The logarithm returns the exponent when given the value of a power.}
\end{aligned}
\]
\end{lessonbox}

\begin{examplebox}{Example 1: Converting between Logarithmic and Exponential Forms}
Write the equivalent log or exponential equation.
\begin{align*}
    \text{a) } \log_2{16} = 4 &\quad \leftrightarrow \quad 2^4 = 16\\
    \text{b) } 7^2 = 49 &\quad \leftrightarrow \quad \log_7{49} = 2\\
    \text{c) } 5^x = 11 &\quad \leftrightarrow \quad \log_5{11} = x\\
    \text{d) } \log_x{42} = 7 &\quad \leftrightarrow \quad x^7 = 42
\end{align*}
\end{examplebox}

\begin{examplebox}{Example 2: Evaluating Logarithms}
Evaluate the following logs:
\begin{align*}
    \text{a) } \log_{10}{100} &= 2\\
    \text{b) } \log_2{64} &= 6\\
    \text{c) } \log_2{\left(\frac{1}{2}\right)} &= -1\\
    \text{d) } \log_3{\left(\frac{1}{27}\right)} &= -3
\end{align*}
\end{examplebox}

\begin{examplebox}{Example 3: Estimating Logarithms}
Estimate \(\log_3{10}\).

The calculator has two log buttons: \(\log\) and \(\ln\) (base 10 and base \(e \approx 2.7183\), respectively).
\end{examplebox}

\begin{examplebox}{Example 4: Using a Calculator to Evaluate Logarithms}
Use a calculator to evaluate, then write an equivalent exponential equation.
\begin{align*}
    \text{a) } \log{52} &\approx 1.716 \quad \leftrightarrow \quad 10^{1.716} = 52\\
    \text{b) } \log{24} &\approx 1.380 \quad \leftrightarrow \quad 10^{1.380} = 24\\
    \text{c) } \ln{12} &\approx 2.485 \quad \leftrightarrow \quad e^{2.485} = 12
\end{align*}
\end{examplebox}

\section{Power Law of Logarithms}

\begin{lessonbox}{Understanding the Power Law of Logarithms}
Evaluate each pair of logarithms on your calculator - What do you notice?
\begin{align*}
    \text{a) } 4 \log{2} &\quad \log{16}\\
    \text{b) } 2 \log{3} &\quad \log{9}\\
    \text{c) } 2 \log{5} &\quad \log{25}
\end{align*}

\textbf{Proof:} Let \( w = \log_b{x} \). Therefore, \( b^w = x \).

\textbf{What use is the power law?}
\begin{enumerate}
    \item It can help simplify logarithmic expressions.
    \item It gives us a method for solving exponential equations.
\end{enumerate}
\end{lessonbox}

\begin{examplebox}{Example 1: Evaluating Logarithms Using the Power Law}
Evaluate the following logarithms:
\begin{align*}
    \text{a) } \log_2{8} &= 3 \quad \text{(Change 8 to } 2^3)\\
    \text{b) } \log_3{\sqrt{27}} &= \frac{3}{2}
\end{align*}
\end{examplebox}

\begin{examplebox}{Example 2: Solving Exponential Equations Using Logarithms}
Solve by first taking the log of both sides and using the power law of logarithms:
\begin{align*}
    \text{a) } 5^t &= 15625 \quad \leftrightarrow \quad t \log{5} = \log{15625} \quad \leftrightarrow \quad t = \frac{\log{15625}}{\log{5}} = 6\\
    \text{b) } 1000 &= 2000(1+0.2)^n \quad \leftrightarrow \quad \frac{1000}{2000} = (1.2)^n \quad \leftrightarrow \quad n = \frac{\log{0.5}}{\log{1.2}} \approx -3.106
\end{align*}
\end{examplebox}

\begin{examplebox}{Example 3: Evaluating Logarithms Using the Change of Base Formula}
Evaluate \(\log_3{54}\):

First, let the expression equal a variable and turn it into the equivalent exponential expression. This actually gives us the change of base formula...
\[
\log_b{m} = \frac{\log{m}}{\log{b}}
\]
\end{examplebox}

\begin{examplebox}{Example 4: Using the Change of Base Formula}
Use the change of base formula to evaluate \(\log_8{254}\):
\[
\log_8{254} = \frac{\log{254}}{\log{8}} \approx 2.9
\]
\end{examplebox}

\begin{examplebox}{Example 5: Solving Compound Interest Problems Using Logarithms}
An investment of \$2000 earns 2\% interest, compounded yearly. A formula to represent this situation is:
\[
A = 2000 (1.02)^n
\]
where \( A \) is the amount of the investment and \( n \) is the number of years of the investment. How long before the investment doubles?
\[
2 = (1.02)^n \quad \leftrightarrow \quad n = \frac{\log{2}}{\log{1.02}} \approx 35
\]
\end{examplebox}

\section{Equivalent Exponential Expressions}

\begin{examplebox}{Example 1: Rewriting Exponentials Using a Different Base}
Rewrite the following using a base of 3.
\[
36 \quad \leftrightarrow \quad 3^6 = (3^2)^3 = 9^3
\]
\end{examplebox}

\begin{examplebox}{Example 2: Using Logarithms to Find the Value of an Exponent}
Use logarithms to find the value of an exponent: \( 3^x = 11 \)
\[
x = \frac{\log{11}}{\log{3}} \approx 2.183
\]
\end{examplebox}
\begin{examplebox}{Example 3: Solve for \( x \) by first writing as powers with the same base.}
\begin{align*}
    27^x &= 9^{2x-3} \implies (3^3)^x = (3^2)^{2x-3} \\ 
    3^{3x} &= 3^{4x-6}  \\
    3x &= 4x-6 x = 6
\end{align*}
\end{examplebox}
\section{Techniques for Solving Exponential Equations}
\begin{lessonbox}{Techniques for Solving Exponential Equations}


Last lesson we solved exponential equations by forcing them to have an equal base and equating their exponents. But what about a situation like this:
\[
5^{2x+4} = 3^{x-7}
\]
\end{lessonbox}

\noindent \textbf{Problem:} We can't simply write them with the same base.\\
\textbf{Solution:} If we take the log of both sides, we can get the variables out of the exponent!

\begin{examplebox}{Example 1: Solve for \( x \)}
\begin{align*}
    \log{5^{2x+4}} &= \log{3^{x-7}}\\
    (2x+4) \log{5} &= (x-7) \log{3}\\
    2x \log{5} + 4 \log{5} &= x \log{3} - 7 \log{3}\\
    2x \log{5} - x \log{3} &= -7 \log{3} - 4 \log{5}\\
    x (2 \log{5} - \log{3}) &= -7 \log{3} - 4 \log{5}\\
    x &= \frac{-7 \log{3} - 4 \log{5}}{2 \log{5} - \log{3}} \approx -4.54
\end{align*}
\end{examplebox}
\begin{examplebox}{Example 2: Solve for \( x \)}
\[
2^x - 2^{-x} = 4
\]
There's no immediate method to solve this, we have to make a couple of adjustments in order to solve. Notice it's somewhat similar to a quadratic equation in that the variables are in decreasing order. We are going to multiply through by \( 2^x \) and this will become more apparent.
\begin{align*}
    2^x (2^x) - 2^x (2^{-x}) &= 4 (2^x)\\
    2^{2x} - 1 &= 4 \cdot 2^x\\
    2^{2x} - 4 \cdot 2^x - 1 &= 0
\end{align*}
Let \( u = 2^x \):
\begin{align*}
    u^2 - 4u - 1 &= 0\\
    u = \frac{4 \pm \sqrt{16 + 4}}{2} = \frac{4 \pm \sqrt{20}}{2} = 2 \pm \sqrt{5}\\
    2^x &= 2 + \sqrt{5}\\
    x &= \log_2 (2 + \sqrt{5}) \approx 1.79
\end{align*}
\end{examplebox}

\begin{examplebox}{Example 3: Using a Half-Life}
An archaeological discovery of an unknown plant fossil has 1/8 the amount of radioactive carbon as plants have today. If the half-life of the carbon is 5730 years, how old is the fossil?
\[
A = A_0 \left(\frac{1}{2}\right)^{\frac{t}{h}}
\]
Given:
\[
\frac{1}{8} = \left(\frac{1}{2}\right)^{\frac{t}{5730}} \quad \leftrightarrow \quad 2^3 = 2^{\frac{t}{5730}} \quad \leftrightarrow \quad 3 = \frac{t}{5730} \quad \leftrightarrow \quad t = 3 \times 5730 = 17190 \text{ years}
\]
\end{examplebox}

\section{Log Laws}

\begin{lessonbox}{Log Laws}
Remember that there were laws for multiplying and dividing powers with the same base. We need to adapt these for use with logarithms.

\[
\log_b(m \cdot n) = \log_b{m} + \log_b{n}
\]
\[
\log_b\left(\frac{m}{n}\right) = \log_b{m} - \log_b{n}
\]
\end{lessonbox}

\begin{examplebox}{Example 1: Simplify using the laws of logarithms}
\begin{align*}
    \text{a) } \log{5} + \log{10} &= \log{(5 \cdot 10)} = \log{50}\\
    \text{b) } \log{12} - \log{2} &= \log{\left(\frac{12}{2}\right)} = \log{6}
\end{align*}
\end{examplebox}

\begin{examplebox}{Example 2: Simplify each expression}
\begin{align*}
    \text{a) } \log{(5a)} + \log{10} - \log{(2b)} &= \log{(5a \cdot 10)} - \log{(2b)} = \log{\left(\frac{50a}{2b}\right)} = \log{\left(\frac{25a}{b}\right)}\\
    \text{b) } \log{x} + 6 \log{y} + 3 \log{z} &= \log{x} + \log{(y^6)} + \log{(z^3)} = \log{(x \cdot y^6 \cdot z^3)}
\end{align*}
\end{examplebox}

\begin{examplebox}{Example 3: Evaluate}
\begin{align*}
    \text{a) } \log{50} + \log{10} - \log{5} &= \log{(50 \cdot 10)} - \log{5} = \log{500} - \log{5} = \log{100} = 2\\
    \text{b) } 4 \log_{12}{2} + 2 \log_{12}{3} &= \log_{12}{(2^4)} + \log_{12}{(3^2)} = \log_{12}{16} + \log_{12}{9} = \log_{12}{144}
\end{align*}
\end{examplebox}

\begin{examplebox}{Example 4: Simplify and state any restrictions}
\begin{align*}
    \text{a) } \log{(2x^2 + 9x - 5)} - \log{(x + 5)} &= \log{\left(\frac{2x^2 + 9x - 5}{x + 5}\right)}\\
    \text{b) } \log{(x+3)} + \log{(2x-5)} &= \log{((x+3)(2x-5))}
\end{align*}
\end{examplebox}

\section{Solving Log Equations}
\begin{lessonbox}{Solving Log Equations}
    
There are three main techniques involved in solving logarithmic equations:
\begin{enumerate}
    \item Use the definition of a logarithm to rewrite the equation as an exponential. Then solve using the techniques for exponential equations.
    \item First simplify using the laws of logarithms, and then rewrite as an exponential to solve.
    \item First simplify using the laws of logarithms and then equate the arguments of the logs on both sides of the equal sign.
\end{enumerate}
\end{lessonbox}

\begin{examplebox}{Example 1: Use the definition to change into an exponential. Solve for \( n \)}
\begin{align*}
\log_3{(n^2 - 3n + 5)} = 2 \\
3^2 = n^2 - 3n + 5 \\
9 = n^2 - 3n + 5 \\
0 = n^2 - 3n - 4 \\
(n-4)(n+1) = 0 \\
n = 4 \text{ or } n = -1
\end{align*}

\end{examplebox}

\begin{examplebox}{Example 2: Simplify and then apply the definition. Solve for \( p \)}
\begin{align*}
\log{(p+5)} - \log{(p+1)} = 3 \implies \log{\left(\frac{p+5}{p+1}\right)} = 3 \\
10^3 = \frac{p+5}{p+1} \\
1000 = \frac{p+5}{p+1}\\
1000(p+1) = p + 5 \\
1000p + 1000 = p + 5 \\
999p = -995\\
p = -\frac{995}{999} \approx -1
\end{align*}

\end{examplebox}

\begin{examplebox}{Example 3: Equating the arguments. Solve for \( x \)}
\begin{align*}
\log{(2x^2 - 7x - 4)} = \log{(2x + 16)} \\
2x^2 - 7x - 4 = 2x + 16 \\
2x^2 - 9x - 20 = 0 \\
(2x + 5)(x - 4) = 0 \\
x = -\frac{5}{2} \text{ or } x = 4
\end{align*}

\end{examplebox}

\section{More Equations}

\begin{examplebox}{Example 1: An exponential with a product in it}
\[
7(1.06^x) = 5.2 \quad \leftrightarrow \quad 1.06^x = \frac{5.2}{7} \quad \leftrightarrow \quad x = \frac{\log{\left(\frac{5.2}{7}\right)}}{\log{1.06}} \approx -5.68
\]
\end{examplebox}

\begin{examplebox}{Example 2: Logarithms where an answer doesn't work out}
\begin{align*}
\log_6{x} + \log_6{(x+1)} = 1 \implies \log_6{[x(x+1)]} = 1 
x(x+1) = 6^1 = 6 \\
x^2 + x - 6 = 0 \\
(x+3)(x-2) = 0 \\
x = -3 \text{ or } x = 2
\end{align*}


Note: \( x = -3 \) is not a valid solution as logarithms of negative numbers are undefined.
\end{examplebox}

\section{The Logarithmic Scale in Science}

\subsection{A. Earthquakes}
\begin{lessonbox}{The Logarithm Scale in the Physical Sciences}
The Richter Scale defines the magnitude of an earthquake as:
\[
M = \log{\left(\frac{I}{I_0}\right)}
\]
Where \( I \) is the earthquake intensity measured and \( I_0 \) is the intensity of a reference quake.
\end{lessonbox}

\begin{examplebox}{Example 1: Comparing Earthquake Magnitudes}
The California earthquake that interrupted the World Series in 1989 measured 6.9 on the Richter scale. The quake that caused the 2004 tsunami in Indonesia measured 9.2. How much more powerful was the Indonesian quake?

Intuitively:
The difference in magnitude is \( 9.2 - 6.9 = 2.3 \).

Using the Definition:
\[
10^{9.2} / 10^{6.9} = 10^{2.3} \approx 200
\]
The Indonesian earthquake was about 200 times more powerful.
\end{examplebox}

\subsection{B. Sound}
The decibel scale compares sound intensities:
\[
L = 10 \log{\left(\frac{I}{I_0}\right)}
\]
Where \( I \) is the intensity of the sound being measured and \( I_0 \) is the threshold of human hearing (the quietest sound we can hear).

\begin{examplebox}{Example 2: Decibel Calculation}
A sound is 5000 times more intense than one that is just audible. How many decibels is the sound?
\[
L = 10 \log{(5000)} \approx 10 \times 3.699 \approx 37
\]
\end{examplebox}

\begin{examplebox}{Example 3: Comparing Loudness}
A jet engine emits a 160 dB sound, while Niagara Falls is 90 dB. How many times louder than Niagara Falls is a jet engine?
\[
10^{160/10} / 10^{90/10} = 10^{16 - 9} = 10^7
\]
A jet engine is \( 10^7 \) times louder than Niagara Falls.
\end{examplebox}

\subsection{C. The pH Scale}
The acidity or alkalinity of a solution is given by the pH scale:
\[
\text{pH} = -\log{[H^+]}
\]
Where \([H^+]\) is the concentration of hydronium ions in the solution in mol/L.

\begin{examplebox}{Example 4: Calculating pH}
The hydronium ions in blood measure at a concentration of \( 4 \times 10^{-7} \) mol/L. What is the pH of blood?
\[
\text{pH} = -\log{(4 \times 10^{-7})} = -(\log{4} + \log{10^{-7}}) = -(\log{4} - 7) \approx -0.602 - 7 = 6.4
\]
\end{examplebox}

\begin{examplebox}{Example 5: Finding Hydronium Ion Concentration}
What is the concentration of hydronium ions in a pool if the pH is 8.2?
\[
8.2 = -\log{[H^+]} \quad \leftrightarrow \quad [H^+] = 10^{-8.2} \approx 6.3 \times 10^{-9} \text{ mol/L}
\]
\end{examplebox}


\end{document}
