Unit 1 - Polynomial Functions 
Unit 2 - Polynomial Equations and inequalities 
Unit 3 - Rational Functions and Equations 
Unit 4 - Radian Measure
Unit 5 - Trig Functions and Equations 
Unit 6 - Log and Exponent Functions and Equations 
Unit 7 - FSE and Exam.

\documentclass{article}
\usepackage[utf8]{inputenc}
\usepackage[T1]{fontenc}
\usepackage{graphicx}
\usepackage{amsmath, amssymb}
\usepackage{xcolor}
\usepackage{tikz}
\usepackage{lipsum}

% Define colors
\definecolor{lessoncolor}{RGB}{74, 144, 226}
\definecolor{examplecolor}{RGB}{92, 184, 92}
\definecolor{notecolor}{RGB}{255, 179, 102}

% Define a command for colorful sections
\newcommand{\colorsection}[1]{\section*{\textcolor{lessoncolor}{#1}}}

% Set up TikZ for graphing
\usetikzlibrary{positioning, arrows.meta, shapes.geometric}

% Document
\begin{document}

\begin{titlepage}
    \centering
    \vspace*{2cm}
    {\LARGE \textcolor{lessoncolor}{Mathematics Lessons}}\par
    \vspace{1cm}
    {\large Your Name}\par
    \vspace{2cm}
    {\large February 20th, 2024}\par
    \vspace{3cm}
\end{titlepage}

\colorsection{Lesson 1: Power Functions}

\textbf{Definition:} A power function is of the form \( f(x) = ax^n \), where \( a \) and \( n \) are constants.

\textbf{Example:} \( f(x) = 2x^3 \) is a power function.

\textbf{Note:} Power functions often exhibit different behaviors based on the sign of \( a \) and the value of \( n \).

\textbf{Exercises:}
\begin{enumerate}
    \item Find the power function for a cubic curve passing through the point \((1, 2)\).
    \item Investigate the behavior of power functions with negative exponents.
\end{enumerate}

\colorsection{Lesson 2: Polynomial Functions}

\textbf{Definition:} A polynomial function is of the form \( P(x) = a_nx^n + a_{n-1}x^{n-1} + \ldots + a_1x + a_0 \), where \( a_i \) are coefficients.

\textbf{Example:} \( P(x) = 3x^4 - 2x^2 + 5 \) is a polynomial function.

\textbf{Note:} The degree of a polynomial is the highest power of the variable with a non-zero coefficient.

\textbf{Exercises:}
\begin{enumerate}
    \item Determine the degree and leading term of the polynomial \( Q(x) = 4x^3 + 2x^2 - 7x + 1 \).
    \item Express the polynomial \( R(x) = x^4 + 3x^2 - 2 \) in standard form.
\end{enumerate}

\colorsection{Lesson 3: Factored Form Polynomial Functions}

\textbf{Definition:} Factored form represents a polynomial as a product of linear factors.

\textbf{Example:} \( P(x) = (x - 2)(x + 1)(x - 3) \) is the factored form of a polynomial.

\textbf{Note:} Factoring helps identify roots and facilitates analysis of the function.

\textbf{Exercises:}
\begin{enumerate}
    \item Factorize the quadratic polynomial \( S(x) = x^2 - 5x + 6 \).
    \item Find the factored form of the cubic polynomial \( T(x) = x^3 + 8 \).
\end{enumerate}

\colorsection{Lesson 4: Transformations of Polynomial Functions}

\textbf{Definition:} Transformations modify the graph of a function.

\textbf{Example:} \( g(x) = f(x - 2) \) shifts the graph of \( f(x) \) two units to the right.

\textbf{Note:} Transformations include shifts, stretches, and reflections.

\textbf{Exercises:}
\begin{enumerate}
    \item Explore the effect of the transformation \( h(x) = f(x + 1) \) on the graph of a polynomial.
    \item Investigate how the graph changes with \( k(x) = -f(x) \).
\end{enumerate}

\colorsection{Lesson 5: Symmetry in Polynomial Functions}

\textbf{Definition:} A function is even if \( f(x) = f(-x) \) for all \( x \) in the domain.

\textbf{Example:} \( h(x) = x^2 \) is an even function.

\textbf{Note:} Odd functions satisfy \( f(x) = -f(-x) \).

\textbf{Exercises:}
\begin{enumerate}
    \item Determine whether the function \( U(x) = x^4 - x^2 \) is even, odd, or neither.
    \item Investigate the symmetry properties of the function \( V(x) = 3x^3 + 2x \).
\end{enumerate}

\colorsection{Graphs and Visuals}

\begin{center}
\begin{tikzpicture}[>=Stealth, scale=0.8]
    % Example graph of a polynomial function
    \draw[->] (-3,0) -- (3,0) node[right] {\(x\)};
    \draw[->] (0,-2) -- (0,4) node[above] {\(f(x)\)};
    \draw[domain=-2.2:2.2, smooth, variable=\x, lessoncolor, thick] plot ({\x},{\x^3 - 2*\x + 1});
    \node[anchor=north east, lessoncolor] at (0,0) {$(0,0)$};
\end{tikzpicture}
\end{center}

\colorsection{Conclusion}

These lessons provide a foundation for understanding power functions, polynomial functions, factored form, transformations, and symmetry in polynomial functions. Visualizing graphs can enhance comprehension and aid in problem-solving.


\newpage
\href{Unit 1}{{https://www.jensenmath.ca/math12af-unit-1}
\end{document}
