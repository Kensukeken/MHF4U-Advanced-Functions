\documentclass{article}
\usepackage[utf8]{inputenc}
\usepackage[T1]{fontenc}
\usepackage{amsmath, amssymb}
\usepackage{xcolor}
\usepackage{graphicx}
\usepackage{lipsum}
\usepackage{enumitem}
\usepackage{framed}
\usepackage{hyperref}
\usepackage[most]{tcolorbox}
\usepackage{pgfplots}

% Define colors
\definecolor{examplecolor}{RGB}{92, 184, 92}
\definecolor{lessoncolor}{RGB}{74, 144, 226}
\definecolor{notecolor}{RGB}{255, 179, 102}

% for small text
\newcommand{\smalltext}[1]{\text{\footnotesize #1}}

\title{Advanced Functions Exam - Part One}
\author{Hia Al Saleh}
\date{May 21st, 2024}

\begin{document}
\maketitle

\subsection*{Part A: Short Answer}
1. List the asymptotes of the following:
\begin{enumerate}
    \item[a)] \( y = \frac{2 + 3x}{6 - x} \)
    \begin{align*}
        \text{V.A: } & \; x-6 = x = 6 \smalltext{ (Denominator equals zero)} \\
        \text{H.A: } & \; y = \frac{2}{-6} = -3 \smalltext{ (Degree of numerator equals degree of denominator, ratio of leading coefficients)}
    \end{align*}
    \item[b)] \( y = \frac{2 + 3x}{x^2 + 5x - 14} \)
    \begin{align*}
        &y = \frac{2 + 3x}{(x + 7)(x - 2)} \smalltext{ (Factor out the denominator)} \\
        \text{V.A: } & \; x = x + 7= -7, x-2= 2 \smalltext{ (Denominator equals zero)} \\
        \text{H.A: } & \; y = 0 \smalltext{ (Degree of denominator is greater than numerator)}
    \end{align*}
    \item[c)] \( y = \frac{x^2 - 5x}{x - 1} \)
    \begin{align*}
        &y = \frac{x(x - 5)}{x - 1} \smalltext{ (Factor out numerator)} \\
        \text{V.A: } & \; x -1 = 1 \smalltext{ (Denominator equals zero)} \\
        \text{O.A: } & \; y = x \smalltext{ (Long division since numerator degree is higher)}
    \end{align*}
\end{enumerate}

2. Solve: \( 5 = \frac{2 + 3x}{6 - x} \)
\begin{align*}
    5(6 - x) &= 2 + 3x & \text{Expand using distributive property} \\
    30 - 5x &= 2 + 3x & \text{Perform multiplication} \\
    30 - 2 &= 5x + 3x & \text{Combine like terms} \\
    28 &= 5x + 3x & \text{Subtract 2 from both sides} \\
    28 &= 8x & \text{Combine like terms} \\
    x &= \frac{28}{8} & \text{Divide both sides by 8} \\
    x &= \frac{7}{2} & \text{Simplify the fraction} \\
\end{align*} 
3. Describe the function \( y=\frac{x^3-5x}{6x^7-4x^3} \) as even, odd or neither.
\begin{align*}
    y(-x) &= \frac{(-x)^3-5(-x)}{6(-x)^7-4(-x)^3} \\
          &= \frac{-x^3+5x}{-6x^7+4x^3} \\
          &= -\frac{x^3-5x}{6x^7-4x^3} \\
          &= -y(x) \\
    \text{Therefore, the function is odd.}
\end{align*}

4. An odd function has 3 vertical asymptotes, one is \( x=3 \). What are the other two?
\begin{align*}
    \text{For an odd function, the asymptotes should be symmetric about the origin.} \\
    \text{Thus, if \( x=3 \) is a vertical asymptote, then the other two are \( x=-3 \) and \( x=0 \).}
\end{align*}

5. True or False? An even non-constant function that is continuous at \( x = 0 \) has a local max or min there.
\begin{align*}
    &\text{True.} \smalltext{ By symmetry, an even function continuous at } x = 0 \smalltext{ has a local extremum at } x = 0.
\end{align*}

6. True or False? A reciprocal function has no roots.
\begin{align*}
    &\text{True.} \smalltext{ A reciprocal function of the form } y = \frac{1}{f(x)} \smalltext{ is undefined where } f(x) = 0.
\end{align*}

7. Convert to radians
\begin{enumerate}
    \item[a)] \( 225^{\circ} \) exact
    \begin{align*}
        225^{\circ} &= 225 \times \frac{\pi}{180} \\
        &= \frac{225\pi}{180} \\
        &= \frac{5\pi}{4}
    \end{align*}
    \item[b)] \( 164^{\circ} \) to 3 decimal places
    \begin{align*}
        164^{\circ} &= 164 \times \frac{\pi}{180} \\
        &= \frac{164\pi}{180} \\
        &= \frac{41\pi}{45} \approx 2.862
    \end{align*}
\end{enumerate}
\newpage
8. Convert to degrees
\begin{enumerate}
    \item[a)] \( \frac{7\pi}{12} \)
    \begin{align*}
        \frac{7\pi}{12} &= \frac{7\pi}{12} \times \frac{180}{\pi} \\
        &= \frac{7 \times 180}{12} \\
        &= 105^{\circ}
    \end{align*}
    \item[b)] \( 2.34 \) (to 1 decimal place)
    \begin{align*}
        2.34 &= 2.34 \times \frac{180}{\pi} \\
        &= \frac{2.34 \times 180}{\pi} \\
        &\approx 134.1^{\circ}
    \end{align*}
\end{enumerate}

9. Determine the angle \( x \in [0, 360^{\circ}] \) and \( \theta \in [0, 2\pi] \)
\begin{enumerate}
    \item[a)] \( \sin x = -0.5 \)
    \begin{align*}
        x &= 210^{\circ}, 330^{\circ} \\
        \theta &= \frac{7\pi}{6}, \frac{11\pi}{6}
    \end{align*}
    \item[b)] \( \cot x = -\sqrt{3} \)
    \begin{align*}
        \cot x &= \frac{\cos x}{\sin x} = -\sqrt{3} \\
        x &= 120^{\circ}, 300^{\circ} \\
        \theta &= \frac{2\pi}{3}, \frac{5\pi}{3}
    \end{align*}
    \item[c)] \( \sec x = 2.5 \) (to 1 decimal place)
    \begin{align*}
        \sec x &= \frac{1}{\cos x} = 2.5 \implies \cos x = 0.4 \\
        x &\approx 66.4^{\circ}, 293.6^{\circ} \\
        \theta &\approx 1.16, 5.12
    \end{align*}
    \item[d)] \( \cos \theta = \frac{1}{\sqrt{2}} \)
    \begin{align*}
        \theta &= \frac{\pi}{4}, \frac{7\pi}{4}
    \end{align*}
    \item[e)] \( \csc \theta = 2 \)
    \begin{align*}
        \csc \theta &= \frac{1}{\sin \theta} \implies \sin \theta = \frac{1}{2} \\
        \theta &= \frac{\pi}{6}, \frac{5\pi}{6}
    \end{align*}
    \item[f)] \( \cot \theta = 2.5 \) (to 1 decimal place)
    \begin{align*}
        \cot \theta &= \frac{1}{\tan \theta} \implies \tan \theta = \frac{1}{2.5} = 0.4 \\
        \theta &\approx 0.38, 3.52
    \end{align*}
\end{enumerate}

10. Solve for the angle \( x \in [0, 360^{\circ}] \) and \( \theta \in [0, 2\pi] \)
\begin{enumerate}
    \item[a)] \( \csc^2 x = 2 \)
    \begin{align*}
        \csc^2 x &= 2 \implies \sin^2 x = \frac{1}{2} \implies \sin x = \pm \frac{1}{\sqrt{2}} \\
        x &= 45^{\circ}, 135^{\circ}, 225^{\circ}, 315^{\circ} \\
        \theta &= \frac{\pi}{4}, \frac{3\pi}{4}, \frac{5\pi}{4}, \frac{7\pi}{4}
    \end{align*}
    \item[b)] \( 3\sec^2 \theta - 4 = 0 \)
    \begin{align*}
        3\sec^2 \theta &= 4 \implies \sec^2 \theta = \frac{4}{3} \implies \cos^2 \theta = \frac{3}{4} \\
        \cos \theta &= \pm \frac{\sqrt{3}}{2} \\
        \theta &= \frac{\pi}{6}, \frac{5\pi}{6}, \frac{7\pi}{6}, \frac{11\pi}{6}
    \end{align*}
    \item[c)] \( \sin 2x = 0.5 \)
    \begin{align*}
        2x &= 30^{\circ}, 150^{\circ}, 390^{\circ}, 510^{\circ} \\
        x &= 15^{\circ}, 75^{\circ}, 195^{\circ}, 255^{\circ} \\
        2x &= \frac{\pi}{6}, \frac{5\pi}{6}, \frac{13\pi}{6}, \frac{17\pi}{6} \\
        x &= \frac{\pi}{12}, \frac{5\pi}{12}, \frac{13\pi}{12}, \frac{17\pi}{12}
    \end{align*}
\end{enumerate}

11. Express as a simple trig function of the angle \( x \).
\begin{enumerate}
    \item[a)] \( \sin \left(\frac{\pi}{2} + x\right) \)
    \begin{align*}
        &= \cos x \smalltext{ (Using co-function identity)}
    \end{align*}
    \item[b)] \( \sec(-x) \)
    \begin{align*}
        &= \sec x \smalltext{ (Even function property of secant)}
    \end{align*}
    \item[c)] \( \tan \left(x - \frac{3\pi}{2}\right) \)
    \begin{align*}
        &= \cot x \smalltext{ (Using period property of tangent)}
    \end{align*}
\end{enumerate}

12. Give the period, amplitude, phase shift and axis of \( y = 5\sin(3x - \pi) - 7 \)
\begin{align*}
    &\text{Period: } \frac{2\pi}{3} \smalltext{ (Coefficient of } x \smalltext{)} \\
    &\text{Amplitude: } 5 \smalltext{ (Coefficient of sine)} \\
    &\text{Phase shift: } \frac{\pi}{3} \smalltext{ (Solving } 3x - \pi = 0 \smalltext{)} \\
    &\text{Axis: } y = -7 \smalltext{ (Vertical shift down 7 units)}
\end{align*}

13. Simplify
\begin{enumerate}
    \item[a)] \( \sin 13 \cos 25 + \sin 25 \cos 13 \)
    \begin{align*}
        &= \sin(13 + 25) \smalltext{ (Using sum-to-product identities)} \\
        &= \sin 38
    \end{align*}
    \item[b)] \( 2\sin 50 \cos 50 \)
    \begin{align*}
        &= \sin 100 \smalltext{ (Using double angle identity for sine)}
    \end{align*}
    \item[c)] \( \cos^2 x - 1 \)
    \begin{align*}
        &= -\sin^2 x \smalltext{ (Using Pythagorean identity)}
    \end{align*}
    \item[d)] \( \cos(\alpha + 2b) \cos(\alpha + b) \sin(\alpha + b) \)
    \begin{align*}
        &\smalltext{Using product-to-sum and trigonometric identities, no further simplification possible without specific angles.}\\
        &= \cos(\alpha + 2b) \left(\frac{1}{2} \sin(2\alpha + 2b)\right) \\
        &= \frac{1}{2} \cos(\alpha + 2b) \sin(2\alpha + 2b)
    \end{align*}
\end{enumerate}

14. Give an exact value for \( \csc 15^{\circ} \)
\begin{align*}
    \csc 15^{\circ} &= \frac{1}{\sin 15^{\circ}} \\
    &= \frac{1}{\sin(45^{\circ} - 30^{\circ})} \\
    &= \frac{1}{\sin 45^{\circ} \cos 30^{\circ} - \cos 45^{\circ} \sin 30^{\circ}} \\
    &= \frac{1}{\frac{\sqrt{2}}{2} \cdot \frac{\sqrt{3}}{2} - \frac{\sqrt{2}}{2} \cdot \frac{1}{2}} \\
    &= \frac{1}{\frac{\sqrt{6} - \sqrt{2}}{4}} \\
    &= \frac{4}{\sqrt{6} - \sqrt{2}} \cdot \frac{\sqrt{6} + \sqrt{2}}{\sqrt{6} + \sqrt{2}} \\
    &= \frac{4(\sqrt{6} + \sqrt{2})}{6 - 2} \\
    &= \sqrt{6} + \sqrt{2}
\end{align*}

15. The population of trout in a river is given by \( N(t) = 1000 + \frac{1000t^2}{t^2 + 100} \), \( t \geq 0 \).
\begin{enumerate}
    \item[a)] What size will the trout population be after a long time?
    \begin{align*}
\lim_{t \rightarrow \infty} N(t) = 1000 + \frac{1000}{1 + \lim_{t \rightarrow \infty} \left[\frac{100}{t^2}\right]} = 1000 + 1000 = 2000 \text{ Trout}
    \end{align*}
    \item[b)] How many trout were in the river to begin with?
    \begin{align*}
        N(0) &= 1000 + \frac{1000 \cdot 0^2}{0^2 + 100} \\
             &= 1000 + 0 \\
             &= 1000
    \end{align*}
    \item[c)] How fast is the trout population growing at three years?
    \begin{align*}
       N(3)=1000+\frac{1000 \cdot 3}{3^2+100}=1000+\frac{9000}{109} \approx 1082.569
    \end{align*}
    \item[d)] What is the average population growth for the first three years?
    \begin{align*}
        \text{Average growth rate} &= \frac{N(3) - N(0)}{3 - 0} \\
        N(3) &= 1000 + \frac{1000 \cdot 3^2}{3^2 + 100} \\
             &= 1000 + \frac{9000}{109} \\
             &= 1000 + 82.57 \\
             &= 1082.57 \\
        \text{Average growth rate} &= \frac{1082.57 - 1000}{3} \\
                                   &= \frac{82.57}{3} \\
                                   &\approx 27.52 \smalltext{ trout per year}
    \end{align*}
\end{enumerate}

\end{document}
