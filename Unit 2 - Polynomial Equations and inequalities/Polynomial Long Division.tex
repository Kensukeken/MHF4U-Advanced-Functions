%&pdflatex
% !TEX encoding = UTF-8 Unicode
% !TEX TS-program = pdflatexmk

% Detecting and warning about obsolete LaTeX commands
% A great document that teaches proper usage is
% http://www.tug.org/teTeX/tetex-texmfdist/doc/latex/general/l2tabuen.pdf
% Get the package from
% http://tug.ctan.org/get/macros/latex/contrib/nag.zip
\RequirePackage[l2tabu, orthodox]{nag}
% among the options are: l2tabu, experimental, abort and orthodox

\documentclass[letterpaper,11pt]{amsart}

% Inhibit use of non-amsmath mathematics
% markup when using amsmath,
% e.g., you cannot use $$ or eqnarray
%\usepackage[all]{onlyamsmath}

% Add hyperlinks
\usepackage[colorlinks=true, pdfstartview=FitV, linkcolor=red, citecolor=red, urlcolor=red]{hyperref}

\usepackage{amssymb}
\usepackage{amsmath}
\usepackage{amsthm}
\usepackage{amsfonts}
\usepackage{multicol}
\usepackage[mathscr]{eucal}
\usepackage{amsmath, amsfonts, amssymb, amstext, amscd, amsthm, graphicx, hyperref, url, enumerate, xspace, esdiff, subfigure}
\usepackage{polynom}
\usepackage[english]{babel}
% Allow to use utf8 encoding
\usepackage[utf8]{inputenc}
\usepackage{enumerate}

\usepackage{ifpdf} 

% Create searchable pdfs
\ifpdf 
\usepackage{mmap}
\fi

% Activate to begin paragraphs with an empty line rather than an indent
%\usepackage[parfill]{parskip}

% allow the inclusion of graphics files
\usepackage{graphicx}

% Automatic line-breaking of displayed math expressions.
%\usepackage{breqn}

\usepackage[thinlines]{easybmat}
% http://www.ctex.org/documents/packages/math/docbmat.pdf

% Next we use \newtheorem to specify our theorem-like environments
% (theorem, definition, etc.) and how to display and number them.
%
% Note: The \theoremstyle declarations affect the appearance of the
% Theorems, Definitions, etc.

\theoremstyle{plain}
\newtheorem{theorem}{Theorem}[section]
\newtheorem{lemma}[theorem]{Lemma}
\newtheorem{corollary}[theorem]{Corollary}
\newtheorem*{claim}{Claim}
\theoremstyle{definition}
\newtheorem{definition}[theorem]{Definition}

\theoremstyle{remark}
\newtheorem{remark}[theorem]{Remark}

% The preamble is also a good place to define new commands and macros.
% This part of the preamble is strictly optional according to your taste.

\usepackage{xspace}


\newcommand{\norm}[1]{\left\| #1 \right\| }
	\newcommand{\limit}[3]{ \lim_{ #1 \to #2 } #3 }
	\newcommand{\summation}[4]{\sum_{ #1 = #2 }^{#3} #4 }
	\newcommand{\Z}{\ensuremath{\mathbb{Z}}\xspace}
	\newcommand{\C}{\ensuremath{\mathbb{C}}\xspace}
	\newcommand{\Q}{\ensuremath{\mathbb{Q}}\xspace}
	\newcommand{\R}{\ensuremath{\mathbb{R}}\xspace}
	\newcommand{\N}{\ensuremath{\mathbb{N}}\xspace}
	\newcommand{\<}{\langle}
	\renewcommand{\>}{\rangle} 
	%\renewcommand{\epsilon}{\varepsilon}
	%\renewcommand{\phi}{\varphi}
	%\newcommand{\x}{{\bf x}}
	%\newcommand{\y}{{\bf y}}
	%\newcommand{\z}{{\bf z}}
	\newcommand{\nil}{\varnothing}
	\newcommand{\A}{{\mathcal A}}
	\newcommand{\F}{{\mathcal F}}
	
\title{Polynomial Long Division}

\date{March 5th, 2024 $\,$  By Kensukeken}

%\usepackage[doublespacing]{setspace}

\begin{document}
\maketitle

Polynomial long division is normal long division but with polynomials instead of just numbers.  It acts in exactly the same ways that our normal quotients of numbers do.  To start with, let's review the process of our usual long division.  We will then extend to polynomials.


\medskip

\textbf{Problem 1:}  Use long division to divide 7 into 323.  

\textbf{Answer:} \longdiv{323}{7}\\

Here we start with the set up $7 \overline{)323}$.  First we ask how many times 7 goes into 3 because 3 is the leading number on the left.  Well...it doesn't, so we move over a decimal place.  Now we ask how many times 7 goes into 32.  We know $7 \cdot 4= 28$, so we throw the 4 on top above the 2.  Then we actually multiply the 4 times 7 and write it underneath the first two terms.  We then subtract this 28 from the 32 and get 4.  Next we just bring the final right hand 3 down to make the 43.  \\

Moving to the next decimal place, we ask how many times 7 goes into the new dividend 43.  We know $7 \cdot 6 =42$, so we can put the 6 on top and the 42 under the 43 and subtract again.  Doing this leaves us with 1.  Since 7 does not divide into 1, we are done.  This means that $323 = 46 \cdot 7 + 1$ where we call 46 the quotient (which the whole number of times 7 can fit into 323), and the remainder 1 (the amount left over after 7 has gone in as many times as possible).


It turns out polynomial long division is very similar.  The algorithm is exactly the same, we just have powers of $x$ to take care of (along with their coefficients).

\textbf{Problem 2:}  Use Polynomial long division to divide $x-1$ into\\ $x^4+x^3-3x^2+3x-2$.

\textbf{Answer:}  We set everything up just like last time.  Start with $$x-1 \overline{)x^4+x^3-3x^2+3x-2}$$  Now we ask the same question every time: what would you multiply $x$ by to get \underline{$\, \qquad $}?   
\newpage
\begin{itemize}
\item \textbf{Step 1:}  What would you multiply $x$ by to get $x^4$?  Of course we would answer $x^3$.  So, put the $x^3$ on top:

\polylongdiv[stage=2]{x^4+x^3-3x^2+3x-2}{x-1}\\

\item \textbf{Step 2:}  Multiply the $x^3$ across both the terms of the divisor and place them under the corresponding powers of $x$ and subtract.  Then bring down the next term just like in normal long division:

\polylongdiv[stage=4]{x^4+x^3-3x^2+3x-2}{x-1}\\

\item \textbf{Step 3:} What would you multiply $x$ by to get $2x^3$?  We would answer $2x^2$, so we add that to the top:

\polylongdiv[stage=5]{x^4+x^3-3x^2+3x-2}{x-1}\\

\item \textbf{Step 4:} Now take the $2x^2$ and multiply across both of the terms of the divisor and place them under the corresponding powers of $x$ and subtract.  Bring down the $3x$ to get:

\polylongdiv[stage=7]{x^4+x^3-3x^2+3x-2}{x-1}\\



\item \textbf{Step 5:} What would you multiply $x$ by to get $-x^2$?  We would answer $-x$, so we add that to the top:

\polylongdiv[stage=8]{x^4+x^3-3x^2+3x-2}{x-1}\\

\newpage

\item \textbf{Step 6:} Now take the $-x$ and multiply across both of the terms of the divisor and place them under the corresponding powers of $x$ and subtract.  Bring down the $-2$ to get:

\polylongdiv[stage=10]{x^4+x^3-3x^2+3x-2}{x-1}

\item \textbf{Step 7:} Finally, what would you multiply $x$ by to get $2x$?  Of course just $2$, so we add that to the top:

\polylongdiv[stage=11]{x^4+x^3-3x^2+3x-2}{x-1}\\


\item \textbf{Step 8:} Now take the $2$ and multiply across both of the terms of the divisor and place them under the corresponding powers of $x$ and subtract. So we get:

\polylongdiv{x^4+x^3-3x^2+3x-2}{x-1}
\end{itemize}

So, $x^4+x^3-3x^2+3x-2 = (x-1) ( x^3+2x^2-x+2)$.  In other words, $x-1$ divides into the polynomial $x^4+x^3-3x^2+3x-2$ ``$x^3+2x^2-x+2$ times".  \\

 \newpage
 
 The next example deals with a remainder.
 
 \textbf{Problem 3:}  Compute the quotient of $3x^4-5x^3+7x^2-9x+3$ by $x^2+2$.
 
 \textbf{Answer:} We set everything up as usual.  Start with $$x^2+2 \overline{)3x^4-5x^3+7x^2-9x+3}$$   

\begin{itemize}
\item \textbf{Step 1:}  What would you multiply the leading term of the divisor, $x^2$, by to get the leading term of the dividend$3x^4$?  We would answer $3x^2$.  So, put the $3x^2$ on top:

\polylongdiv[stage=2]{3x^4-5x^3+7x^2-9x+3}{x^2+2}\\

\item \textbf{Step 2:}  Multiply the $3x^2$ across both the terms of the divisor and place them under the corresponding powers of $x$ and subtract.  Then bring down the next term just like in normal long division.  Notice that we get something a little different this time.  But different isn't bad, we can just continue as normal:

\polylongdiv[stage=4]{3x^4-5x^3+7x^2-9x+3}{x^2+2}\\

\item \textbf{Step 3:} What would you multiply $x^2$ by to get $-5x^3$?  We would answer $-5x$, so we add that to the top:

\polylongdiv[stage=5]{3x^4-5x^3+7x^2-9x+3}{x^2+2}\\

\item \textbf{Step 4:} Now take the $-5x$ and multiply across both of the terms of the divisor and place them under the corresponding powers of $x$ and subtract.  Bring down the $3$ to get:

\polylongdiv[stage=7]{3x^4-5x^3+7x^2-9x+3}{x^2+2}\\



\item \textbf{Step 5:} What would you multiply $x^2$ by to get $x^2$?  It sounds stupid, but it still makes sense so we'll answer it.  We'd have to multiply by $1$, so we add that to the top:

\polylongdiv[stage=8]{3x^4-5x^3+7x^2-9x+3}{x^2+2}\\


\item \textbf{Step 6:} Now take the $1$ and multiply across both of the terms of the divisor and place them under the corresponding powers of $x$ and subtract:

\polylongdiv[stage=10]{3x^4-5x^3+7x^2-9x+3}{x^2+2}\\

\item \textbf{Step 7:} What would we multiply $x^2$ by to get $x$?  Now that is a weird question.  Notice we would have to multiply by $\frac{1}{x}$, which is not in the same pattern we've been doing.  So we stop here.  Essentially we are only multiplying by nonnegative powers of $x$.  If we ever hit a negative power of $x$ we know we are done.  But there's still that $x+1$ sitting down there.  Just like the 1 that was left over in problem 1, this is the remainder for this problem.   

So, $3x^4-5x^3+7x^2-9x+3 = (x^2+2)(3x^2-5x+1) + x+1$.  In other words, we get that $x^2+2$ divides into $3x^4-5x^3+7x^2-9x+3$ a total ``number"  of $3x^2-5x+1$ times with a remainder of $x+1$.

\end{itemize}
\bigskip

\textbf{\underline{\Large{The Theory}}}

Ok, where the heck is this stuff coming from?  Let's start with the normal long division.   Division really comes from the process to satisfy the following theorem.\\

\textbf{Theorem (The Division Algorithm):}  For any integers $a,b$ with $b \neq 0$, there exist \textbf{unique} integers $q$ and $r$ such that $a=bq+r$ where $0 \leq r < |b|$.  The integer $q$ is called the quotient and $r$ is the remainder.\\

The process of long division is how we find the $q$ and $r$ to put in the equation.  When we do polynomial long division, it is almost identical but we use functions instead (it even goes by the same name).\\

\textbf{Theorem (Division Algorithm):}  Let $D$ be an integral domain, and $a(x), b(x)$ be in $D[x]$ such that the leading coefficients of $a(x)$ is invertible in $D$.  Then there exist polynomials $q(x), r(x)$ in $D[x]$ such that $$b(x) = a(x) q(x) + r(x) $$ where $r(x) =0$ or has a degree less than that of $a(x)$.  Furthermore, polynomials $q(x)$ and $r(x)$ are unique.\\

Don't worry about the $D[x]$ and invertible stuff, it just means we are working with polynomials and you can divide by them.   What we get out of this is that it looks identical to the usual case with numbers, and that the process of polynomial long divison gives us the quotient $q(x)$ and remainder $r(x)$.
\end{document}  
